\documentclass{article}[12pt]
\usepackage{xepersian}
\usepackage{setspace}

\renewcommand{\baselinestretch}{1.5} 
\begin{document}
\begin{center}
به نام خدا\\
گزارش تمرین سودوکو\\
ایمان تبریزیان\\
۹۳۳۱۰۳۲

\end{center}
در این گزارش توضیحاتی درباره‌ی تخمین زمانی الگوریتم توسط مونت کارلو ارائه می‌گردد:\\
لازم به ذکر است که در تمامی حالت‌ها اندازه‌ی جدول ما ۴*۴ است.\\
پارامتر مورد بررسی:\\
الف) تاثیر تعداد اعداد مشخص شده در مساله قابل حل:\\
\begin{tabular}{|c|c|c|c|c|}
\hline
تعداد اعداد مشخض شده & ۱ & ۲ & ۳ & ۴\\
\hline
عدد خروجی از مونت کارلو & ۶۵ & ۲۱ & ۲۱ & ۵\\

\hline

\end{tabular}\\


هماگونه که انتظار داشتیم با افزایش تعداد اعداد مشخص شده در جدول تعداد گره‌های مورد محاسبه کاهش می‌یابد این نتیجه نیز همان‌طور که انتظار داشتیم بود.\\
ب) تاثیر تعداد اعداد مشخص شده در مساله غیر قابل حل:\\
\begin{tabular}{|c|c|c|c|c|}
\hline
تعداد اعداد مشخض شده & ۲ & ۳ & ۴ & ۵\\
\hline
عدد خروجی از مونت کارلو & ۶۵ & ۲۱ & ۲۱ & ۶۵\\

\hline

\end{tabular}\\
همان‌طور که انتظار داشتیم چون مساله غیر قابل حل است هیچ یک گره‌ها به جواب نمی‌رسد و تقریبا باید تمامی گره‌های درخت پردازش شود تا جواب مطلوب حاصل گردد.\\
ج) تاثیر تعداد اعداد مشخص شده در مساله طراحی:۲۶۱\\
چون در این حالت تمامی اعداد ۰ صفر است، فضای حالت ما بسیار بزرگ است و بنابراین، گره‌های پردازش شده بسیار بیشتر است.\\
لازم به ذکر است که از قسمت‌های امتیازی نیز، قسمت متغیر بودن اندازه جدول را نیز زده‌ام.
\end{document}
